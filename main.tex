% LTeX: language=pt-BR

\documentclass[10pt,oneside,onecolumn]{article}
% LTeX: language=pt-BR

\usepackage[brazilian]{babel} % Regras tipográficas
\usepackage{graphicx} % Required for inserting images
\usepackage[hmarginratio=1:1,top=32mm,columnsep=21pt]{geometry} % Margens do documento
\usepackage{amsmath}
\usepackage{amssymb}
\usepackage{float}
\usepackage{multicol}
\usepackage{hyperref}
\usepackage{lipsum}
\usepackage{caption}
\usepackage{subcaption}
\usepackage[dvipsnames]{xcolor} % Para a definição de cores
\usepackage[
	cachedir = \detokenize{.cache/minted}
]{minted}   % Para a inclusão literal de arquivos com sintaxe

\usepackage[
	style=abnt,
	backend=biber,
	language=brazilian
]{biblatex}

\addbibresource{references.bib}

\usepackage{siunitx} % Facilita o uso das unidades do SI
	\sisetup{output-decimal-marker = {,}} % Configura a vírgula como o separador decimal
	\sisetup{range-phrase = \text{--}}

\usepackage[hmarginratio=1:1,top=32mm,columnsep=21pt]{geometry} % Margens do documento
% \usepackage[hang, small,labelfont=bf,up,textfont=it,up]{caption} % Legendas customizadas pra tabelas e imagens
\usepackage{booktabs} % Tabelas variáveis

\usepackage[sc]{mathpazo} % Usando uma fonte diferente para o documento
	\usepackage[T1]{fontenc} % Use 8-bit encoding that has 256 glyphs
	\linespread{1.05} % Aumentando o espaçamento entre as linhas (a fonte não fica tão legal com o espaçamento padrão)
	\usepackage{microtype} % Não lembro o que isso faz

% Configuração das fontes
\usepackage{fontspec}
	\setmainfont{TeX Gyre Pagella} % Palatino clone
	\setmonofont{BlexMonoNerdFontPropo}[Scale=0.75]
	% Configuração de fallback para emojis
	\newfontfamily{\emojifont}{Noto Color Emoji}[Renderer=HarfBuzz,Scale=0.75]
	\usepackage{newunicodechar}
	\newunicodechar{🐍}{{\emojifont 🐍}}
	\newunicodechar{🦀}{{\emojifont 🦀}}

\usepackage{enumitem} % Listas customizadas
	\setlist[itemize]{noitemsep} % Pra tornar as listas mais compactas

\usepackage{abstract}
	\renewcommand{\abstractnamefont}{\normalfont\bfseries} % Deixa o "Resumo" em negrito
	% \renewcommand{\abstracttextfont}{\normalfont\small\itshape} % Deixa o conteúdo do resumo em itálico
	\renewcommand{\abstracttextfont}{\normalfont\small}

\usepackage{titlesec} % Customização do título
	\renewcommand\thesection{\Roman{section}} % Números romanos para as secções
	\renewcommand\thesubsection{\roman{subsection}} % Para as subsecções também
	\titleformat{\chapter}[display]{\normalfont\huge\bfseries}{\chaptertitlename\ \thechapter}{20pt}{\Huge} % Formato dos capítulos
	\titlespacing*{\chapter}{0pt}{0pt}{20pt} % Reduz drasticamente a margem superior dos capítulos (era ~50pt por padrão)
	\titleformat{\section}[block]{\large\scshape}{\thesection.}{1em}{} % Muda a aparência do título das secções
	\titleformat{\subsection}[block]{\large}{\thesubsection.}{1em}{} % Muda a aparência do título das subsecções

\usepackage{fancyhdr} % Cabeçalho
	\pagestyle{fancy} % Cabeçalho em todas as páginas
	\fancyhead{}
	\fancyfoot{}
	\fancyhead[L]{Linguagens de Programação}% $\bullet$ Lucca Pellegrini $\bullet$ \today} % Custom header text
	\fancyhead[C]{\textbf{Rust}}% $\bullet$ Lucca Pellegrini $\bullet$ \today} % Custom header text
	\fancyhead[R]{PUC Minas}% $\bullet$ Lucca Pellegrini $\bullet$ \today} % Custom header text
	\fancyfoot[C]{\thepage} % Custom footer text

\usepackage{titling} % Customização do título

\usepackage{url} % Pra ajudar a lidar com urls chatos

\usepackage{amsmath, amsthm, amssymb, amsfonts} % AMS-TeX pra equações (em geral) mais bonitas e pra umas outras coisas
\usepackage{csquotes}
\usepackage{svg}

\usepackage{enotez}
	\setenotez{list-name={Notas}, backref=true}


\title{{\huge\bfseries Rust: Uma Análise da Linguagem}\\
Linguagens de Programação}

\author{ % Autores
    \textsc{Amanda Canizela Guimarães} \\
    \normalsize{\href{mailto:amanda.canizela@gmail.com}{\texttt{amanda.canizela@gmail.com}}}
    \and
    \textsc{Ariel Inácio Jordão Coelho} \\
    \normalsize{\href{mailto:arielijordao@gmail.com}{\texttt{arielijordao@gmail.com}}}
    \and
    \textsc{Lucca Pellegrini} \\
    \normalsize{\href{mailto:lucca@verticordia.com}{\texttt{lucca@verticordia.com}}}
}

\date{\today}

% =========================
% Documento
% =========================
\begin{document}

\maketitle

\begin{abstract}

    Esse trabalho aprofunda os conhecimentos sobre a linguagem de programação
    Rust. Nele, será apresentada a história de tal código, junto da criação de
    um panorama da mesma, colaborando para um melhor entendimento de suas
    aplicações e utilidades, tanto atuais quanto ao longo da história. Além
    disso, o grupo trará exemplos para realizar uma análise de sua
    implementação, mostrando o uso da linguagem e seus paradigmas diante de
    práticas cotidianas e mais avançadas. Por fim, será posto em evidência a
    importância da linguagem em questão para a fomentação de outras e seu
    impacto na história do desenvolvimento tecnológico.

\end{abstract}

\section{Introdução}

O primeiro parágrafo de cada seção não deve ser indentado. Este modelo segue as
especificações oficiais da SBC para artigos submetidos.

Os parágrafos subsequentes devem ter uma indentação de 1,27 cm no início da
primeira linha, conforme as regras estabelecidas.

\section{Panorama geral}

Nesta seção será apresentada uma visão geral sobre a história da linguagem
Rust, suas influências e o contexto de seu surgimento no cenário da programação
moderna.

\section{Aplicações e mercado}

Aqui serão discutidas as principais áreas de aplicação da linguagem Rust, bem
como sua presença no mercado de tecnologia e em projetos de código aberto.

\section{Classificação e Paradigmas}

\subsection{Introdução}

% TODO: revisar se essa introdução não repete muito do que será dito acima

Rust é uma linguagem compilada de propósito geral orientada ao desenvolvimento
de sistemas---isso é, orientada àquelas aplicações em que o desempenho do
programa é altamente valorizado---geralmente compreendida como uma alternativa
moderna e funcional a C++ que minimiza os riscos associados à gerência manual
de memória, por meio de uma variedade de abstrações, sem sacrificar desempenho
\cite[Prefácio]{blandy2021programming}. A linguagem segue a mesma ambição
proposta pelo autor de C++:

\begin{quote}
    Em geral, implementações de C++ seguem o princípio do
    \textit{zero-overhead}: você não paga pelo que não usa. E, além disso,
    aquilo que você usa não poderia ser implementado manualmente de forma
    melhor. \cite{stroustrup2004abstraction} (tradução livre)
\end{quote}

Apesar disso, o Rust não se limita apenas à programação de sistemas
operacionais, sistemas embarcados, e outras aplicações de \textit{baixo nível}:
suas características ergonômicas e sua flexibilidade possibilitam o
desenvolvimento de servidores web, ferramentas para \textit{DevOps}, interfaces
gráficas, aplicações web \textit{back end} e---por meio de \textit{web
assembly}---\textit{front end}, jogos digitais, bancos de dados, compiladores,
aplicativos móveis, análise e transcodificação de multimídia, criptomoedas,
programas assíncronos, aplicações de bioinformática, ferramentas de busca,
sistemas \textsc{IoT}, aprendizado de máquina, navegadores, \textit{et cetera}.
Seu ecossistema, já amadurecido, inclui uma variedade de bibliotecas para os
desenvolvedores, além de um ferramental integrado, que incorpora servidores
\textit{Language Server Protocol} (\textsc{lsp}) para integração com
\textsc{ide}s, formatadores de código, e resolução automática e controle de
versões de dependências. \cite[Introdução]{klabnik2021rust}

% TODO: revisar se o parágrafo abaixo é realmente necessário...

Nesta seção, exploraremos os fundamentos da sintaxe e dos paradigmas que
definem o Rust através de exemplos práticos de código. O objetivo não é
fornecer um tutorial completo ou exaustivo sobre a linguagem, mas ilustrar seus
principais conceitos e mecanismos por meio de exemplos sintaticamente corretos
e compiláveis na versão mais recente da linguagem. Abordaremos desde a sintaxe
básica---declaração de variáveis, tipos primitivos e compostos---até conceitos
mais avançados como o sistema de \textit{ownership}, \textit{borrowing}, e
\textit{lifetimes}, que constituem o coração da proposta inovadora do Rust para
gerenciamento seguro de memória. Também discutiremos brevemente as
características funcionais da linguagem, como \textit{pattern matching},
\textit{iterators}, \textit{closures} e programação genérica. É importante
ressaltar que os exemplos apresentados são ilustrativos e didáticos, destinados
a demonstrar a sintaxe e os conceitos fundamentais, e não constituem
necessariamente código de produção. Para um aprendizado aprofundado sobre
\textit{como} programar em Rust e aplicar seus conceitos em projetos reais,
recomenda-se fortemente a consulta às referências bibliográficas citadas ao
longo do texto, em especial os livros \cite{klabnik2021rust} e
\cite{blandy2021programming}, que oferecem cobertura abrangente e didática da
linguagem.

\subsection{Sintaxe Básica}

A grosso modo, a linguagem dispõe de uma sintaxe familiar e \textit{C-like}, e
segue um paradigma \textit{imperativo e estruturado}, com certas
características procedurais clássicas, e outras funcionais modernas. A tipagem
é estática e extremamente rígida, para providenciar segurança a tempo de
compilação, mas o compilador é capaz de inferir o tipo de uma variável usando
\textit{inferência bidirecional} \cite{pierce2000local}. No exemplo abaixo,
usamos a palavra \mintinline{rust}{fn} para declarar uma função
\mintinline{rust}{main}, que não recebe parâmetros, e marca o ponto de entrada
do programa---nela, usamos a palavra \mintinline{rust}{let} para declarar uma
variável \texttt{x}, e em seguida o macro \mintinline{rust}{println!()} exibe o
valor dessa variável na saída padrão.\footnote{O ponto de exclamação indica que
\mintinline{rust}{println!()} não é uma função, mas um macro: no Rust, muitas
funcionalidades são implementadas por meio de metaprogramação com macros
sofisticados, que permitem que o compilador otimize ao máximo até mesmo as
operações mais simples. Nesse caso, o macro é executado a tempo de compilação,
e é o \textit{compilador} que gera o código que converte nosso inteiro para uma
string a ser exibida---ou seja, é muito mais eficiente que um
\texttt{printf()} em uma linguagem tradicional, que faz a conversão a tempo de
execução. \cite[cap.~7]{gjengset2021rust}} O tipo de \texttt{x} é inferido como
\mintinline{rust}{i32}, um número inteiro de 32 bits, com sinal.\footnote{Há
uma variedade de tipos escalares padrão, mas em geral, tipos numéricos têm
tamanhos explícitos: \mintinline{rust}{u16} é um \textit{unsigned} de 16 bits,
\mintinline{rust}{u8} é um \textit{byte} individual, \mintinline{rust}{f64} é
um \textit{float} de 64 bits, \mintinline{rust}{isize} é um \textit{signed}
cujo tamanho é igual ao tamanho de uma palavra na arquitetura atual, e assim
por diante.}

\begin{minted}{rust}
    fn main() {
        let x = 5;
        println!("O valor de ‘x’ é: {x}");
    }
\end{minted}

Todas as variáveis são imutáveis por padrão. Para modificá-las, é preciso
declará-las mutáveis com a palavra \mintinline{rust}{mut}. Também é possível
especificar o tipo da variável após seu  nome na declaração, como se faz em
Python ou TypeScript. No exemplo abaixo, declaramos a variável \textit{mutável}
\texttt{c} com tipo \mintinline{rust}{char}---isso é, um caractere
\textsc{utf}-8:\footnote{Para ser mais exato, um \mintinline{rust}{char} é um
\textit{code point} especificado pelo padrão \textsc{utf}-8 \cite[§1]{rfc3629},
e não um \textit{grapheme cluster} do padrão Unicode \cite[§3]{uax29}. A
distinção é sutil, mas importante para internacionalização da linguagem.
\cite[cap.~3, §3]{blandy2021programming}}

\begin{minted}{rust}
    fn main() {
        let mut c: char = '🐍';
        println!("O valor de ‘c’ inicialmente é: {c}");
        c = '🦀'; // Modificar a variável só é possível pois usamos ‘mut’ acima.
        println!("O valor de ‘c’ modificado é: {c}");
    }
\end{minted}

Isso vale não somente para os tipos primitivos, mas também para os tipos
compostos. Há três tipos compostos padrão que podem estar na \textit{stack}:
tuplas, arrays, e slices. Uma tupla é um agrupamento composto de valores de
tipos variados com tamanho fixo---uma vez criada, o seu tamanho não pode mudar.
Elas podem ser declaradas listando seus itens, separados por vírgulas, entre
parênteses, e seus elementos podem ser acessados com um ponto (\texttt{.})
seguido de um índice numérico, ou por desestruturação. Já um array é um
agrupamento contíguo de elementos do mesmo tipo; seu tamanho também é fixo, e
seus elementos são acessados com um índice numérico entre colchetes, com
verificação do índice sendo feita automaticamente para impedir acessos
\textit{out of bounds}. Uma slice, por outro lado, é um tipo de
\textit{referência} (que em Rust chamamos de \textit{empréstimo} ou
\textit{borrow}) \cite[cap.~16, §5]{google2024comp}, declarada prefixando o
tipo com o caractere \texttt{\&}, que aponta para uma sequência contígua de
elementos.\footnote{A fim de exemplo, podemos declarar uma referência a um tipo
primitivo assim: \mintinline{rust}{let b: &i32 = &a}. Com isso, declaramos uma
variável \texttt{b}, de tipo \mintinline{rust}{&i32} (isso é, uma referência ou
\textit{empréstimo} de um inteiro de 32 bits com sinal), que aponta para uma
variável \texttt{a}, de tipo \mintinline{rust}{i32}. Por outro lado, se
declaramos \mintinline{rust}{let b: &[i32] = &a}, temos uma \textit{slice} de
um array de \mintinline{rust}{i32}s: um empréstimo de um segmento contíguo da
memória que contém zero ou mais elementos do tipo \mintinline{rust}{i32}. O
\textit{borrow checker} do Rust, que discutiremos mais adiante, assegura que o
empréstimo seja válido e impede que a slice seja usada se o dado referenciado
deixar de existir---essa garantia é feita a tempo de compilação. Diferentemente
de uma referência simples, a slice também carrega informações de comprimento
(\texttt{len}) e capacidade.} No exemplo a seguir, uma tupla contendo um
inteiro, uma \textit{string slice},\footnote{Mais adiante, explicaremos os
diferentes tipos de strings em Rust, para esclarecer por quê e quando usamos
\mintinline{rust}{String}, \mintinline{rust}{&str}, \mintinline{rust}{CStr},
\mintinline{rust}{OsStr}, \mintinline{rust}{OsString}, \mintinline{rust}{Path},
\mintinline{rust}{PathBuf}, \textit{et cetera}.} e um booleano é declarada,
acessada e desestruturada; abaixo, criamos um array mutável de
\textit{doubles}, alteramos um elemento, e depois criamos uma slice
não-mutável, que referencia os elementos da segunda à quarta posição no array
(índices 1 a 3):

\begin{minted}{rust}
    fn main() {
        // Criamos uma tupla, com três elementos de diferentes tipos
        let minha_tupla: (i32, &str, bool) = (47, "Hello, World!", false);
        println!("Item na posição 1 da tupla: {}", minha_tupla.1);

        // Desestruturamos a tupla, criando três variáveis individuais
        let (num, string, booleano) = minha_tupla;
        println!("Variáveis: ({}, {}, {})", num, string, booleano);

        // Criamos um array mutável com capacidade para cinco floats de 64 bits
        let mut meu_array: [f64; 5] = [0.1, 1.0, 1.5, 7.25, 1e100];
        meu_array[0] = 3.14;
        println!("Item no índice 0 do array: {}", meu_array[0]);

        // Criamos uma referência imutável que ‘enxerga’ os índices 1 a 3 do array
        let minha_slice: &[f64] = &meu_array[1..4];
        println!("Item no índice 2 da slice: {}", minha_slice[2]);
    }
\end{minted}

\section{Análise crítica}

Esta seção apresentará uma análise crítica da linguagem Rust, destacando três
pontos positivos e três pontos negativos observados a partir de seu uso e das
avaliações da comunidade.

\section{Legado e influência}

Será abordado o legado deixado pelo Rust, bem como as linguagens, ferramentas e
práticas de desenvolvimento que foram influenciadas por ele.

\section{Conclusão}

A seção final sintetiza os principais pontos discutidos no trabalho, reforçando
a relevância da linguagem Rust no cenário atual da computação.

\clearpage
\printbibliography

\end{document}

% vim: et ts=4 sts=4 sw=4
