% LTeX: language=pt-BR

\documentclass[10pt,oneside,onecolumn]{article}
% LTeX: language=pt-BR

\usepackage[brazilian]{babel} % Regras tipográficas
\usepackage{graphicx} % Required for inserting images
\usepackage[hmarginratio=1:1,top=32mm,columnsep=21pt]{geometry} % Margens do documento
\usepackage{amsmath}
\usepackage{amssymb}
\usepackage{float}
\usepackage{multicol}
\usepackage{hyperref}
\usepackage{lipsum}
\usepackage{caption}
\usepackage{subcaption}
\usepackage[dvipsnames]{xcolor} % Para a definição de cores
\usepackage[
	cachedir = \detokenize{.cache/minted}
]{minted}   % Para a inclusão literal de arquivos com sintaxe

\usepackage[
	style=abnt,
	backend=biber,
	language=brazilian
]{biblatex}

\addbibresource{references.bib}

\usepackage{siunitx} % Facilita o uso das unidades do SI
	\sisetup{output-decimal-marker = {,}} % Configura a vírgula como o separador decimal
	\sisetup{range-phrase = \text{--}}

\usepackage[hmarginratio=1:1,top=32mm,columnsep=21pt]{geometry} % Margens do documento
% \usepackage[hang, small,labelfont=bf,up,textfont=it,up]{caption} % Legendas customizadas pra tabelas e imagens
\usepackage{booktabs} % Tabelas variáveis

\usepackage[sc]{mathpazo} % Usando uma fonte diferente para o documento
	\usepackage[T1]{fontenc} % Use 8-bit encoding that has 256 glyphs
	\linespread{1.05} % Aumentando o espaçamento entre as linhas (a fonte não fica tão legal com o espaçamento padrão)
	\usepackage{microtype} % Não lembro o que isso faz

% Configuração das fontes
\usepackage{fontspec}
	\setmainfont{TeX Gyre Pagella} % Palatino clone
	\setmonofont{BlexMonoNerdFontPropo}[Scale=0.75]
	% Configuração de fallback para emojis
	\newfontfamily{\emojifont}{Noto Color Emoji}[Renderer=HarfBuzz,Scale=0.75]
	\usepackage{newunicodechar}
	\newunicodechar{🐍}{{\emojifont 🐍}}
	\newunicodechar{🦀}{{\emojifont 🦀}}

\usepackage{enumitem} % Listas customizadas
	\setlist[itemize]{noitemsep} % Pra tornar as listas mais compactas

\usepackage{abstract}
	\renewcommand{\abstractnamefont}{\normalfont\bfseries} % Deixa o "Resumo" em negrito
	% \renewcommand{\abstracttextfont}{\normalfont\small\itshape} % Deixa o conteúdo do resumo em itálico
	\renewcommand{\abstracttextfont}{\normalfont\small}

\usepackage{titlesec} % Customização do título
	\renewcommand\thesection{\Roman{section}} % Números romanos para as secções
	\renewcommand\thesubsection{\roman{subsection}} % Para as subsecções também
	\titleformat{\chapter}[display]{\normalfont\huge\bfseries}{\chaptertitlename\ \thechapter}{20pt}{\Huge} % Formato dos capítulos
	\titlespacing*{\chapter}{0pt}{0pt}{20pt} % Reduz drasticamente a margem superior dos capítulos (era ~50pt por padrão)
	\titleformat{\section}[block]{\large\scshape}{\thesection.}{1em}{} % Muda a aparência do título das secções
	\titleformat{\subsection}[block]{\large}{\thesubsection.}{1em}{} % Muda a aparência do título das subsecções

\usepackage{fancyhdr} % Cabeçalho
	\pagestyle{fancy} % Cabeçalho em todas as páginas
	\fancyhead{}
	\fancyfoot{}
	\fancyhead[L]{Linguagens de Programação}% $\bullet$ Lucca Pellegrini $\bullet$ \today} % Custom header text
	\fancyhead[C]{\textbf{Rust}}% $\bullet$ Lucca Pellegrini $\bullet$ \today} % Custom header text
	\fancyhead[R]{PUC Minas}% $\bullet$ Lucca Pellegrini $\bullet$ \today} % Custom header text
	\fancyfoot[C]{\thepage} % Custom footer text

\usepackage{titling} % Customização do título

\usepackage{url} % Pra ajudar a lidar com urls chatos

\usepackage{amsmath, amsthm, amssymb, amsfonts} % AMS-TeX pra equações (em geral) mais bonitas e pra umas outras coisas
\usepackage{csquotes}
\usepackage{svg}

\usepackage{enotez}
	\setenotez{list-name={Notas}, backref=true}


\title{{\huge\bfseries Rust: Uma Análise da Linguagem}\\
Linguagens de Programação}

\author{ % Autores
	\textsc{Amanda Canizela Guimarães} \\
	\normalsize{\href{mailto:amanda.canizela@gmail.com}{\texttt{amanda.canizela@gmail.com}}}
	\and
	\textsc{Ariel Inácio Jordão Coelho} \\
	\normalsize{\href{mailto:arielijordao@gmail.com}{\texttt{arielijordao@gmail.com}}}
	\and
	\textsc{Lucca Pellegrini} \\
	\normalsize{\href{mailto:lucca@verticordia.com}{\texttt{lucca@verticordia.com}}}
}

\date{\today}

% =========================
% Documento
% =========================
\begin{document}

\maketitle

\begin{abstract}

	Esse trabalho aprofunda os conhecimentos sobre a linguagem de programação
	Rust\cite{klabnik2021rust}. Nele, será apresentada a história de tal
	código, junto da criação de um panorama da mesma, colaborando para um
	melhor entendimento de suas aplicações e utilidades, tanto atuais quanto ao
	longo da história. Além disso, o grupo trará exemplos para realizar uma
	análise de sua implementação, mostrando o uso da linguagem e seus
	paradigmas diante de práticas cotidianas e mais avançadas. Por fim, será
	posto em evidência a importância da linguagem em questão para a fomentação
	de outras e seu impacto na história do desenvolvimento tecnológico.

\end{abstract}

\section{Introdução}

O primeiro parágrafo de cada seção não deve ser indentado. Este modelo segue as
especificações oficiais da SBC para artigos submetidos.

Os parágrafos subsequentes devem ter uma indentação de 1,27 cm no início da
primeira linha, conforme as regras estabelecidas.

\section{Panorama geral}

Nesta seção será apresentada uma visão geral sobre a história da linguagem
Rust, suas influências e o contexto de seu surgimento no cenário da programação
moderna.

\section{Aplicações e mercado}

Aqui serão discutidas as principais áreas de aplicação da linguagem Rust, bem
como sua presença no mercado de tecnologia e em projetos de código aberto.

\section{Classificação e paradigmas}

Serão explicados os paradigmas de programação adotados pelo Rust, com exemplos
práticos que ilustram suas principais características e diferenças em relação a
outras linguagens.

\section{Análise crítica}

Esta seção apresentará uma análise crítica da linguagem Rust, destacando três
pontos positivos e três pontos negativos observados a partir de seu uso e das
avaliações da comunidade.

\section{Legado e influência}

Será abordado o legado deixado pelo Rust, bem como as linguagens, ferramentas e
práticas de desenvolvimento que foram influenciadas por ele.

\section{Conclusão}

A seção final sintetiza os principais pontos discutidos no trabalho, reforçando
a relevância da linguagem Rust no cenário atual da computação.

\printbibliography

\end{document}
