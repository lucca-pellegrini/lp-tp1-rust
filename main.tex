% LTeX: language=pt-BR

\documentclass[10pt,oneside,onecolumn]{report}
% LTeX: language=pt-BR

\usepackage[brazilian]{babel} % Regras tipográficas
\usepackage{graphicx} % Required for inserting images
\usepackage[hmarginratio=1:1,top=32mm,columnsep=21pt]{geometry} % Margens do documento
\usepackage{amsmath}
\usepackage{amssymb}
\usepackage{float}
\usepackage{multicol}
\usepackage{hyperref}
\usepackage{lipsum}
\usepackage{caption}
\usepackage{subcaption}
\usepackage[dvipsnames]{xcolor} % Para a definição de cores
\usepackage[
	cachedir = \detokenize{.cache/minted}
]{minted}   % Para a inclusão literal de arquivos com sintaxe

\usepackage[
	style=abnt,
	backend=biber,
	language=brazilian
]{biblatex}

\addbibresource{references.bib}

\usepackage{siunitx} % Facilita o uso das unidades do SI
	\sisetup{output-decimal-marker = {,}} % Configura a vírgula como o separador decimal
	\sisetup{range-phrase = \text{--}}

\usepackage[hmarginratio=1:1,top=32mm,columnsep=21pt]{geometry} % Margens do documento
% \usepackage[hang, small,labelfont=bf,up,textfont=it,up]{caption} % Legendas customizadas pra tabelas e imagens
\usepackage{booktabs} % Tabelas variáveis

\usepackage[sc]{mathpazo} % Usando uma fonte diferente para o documento
	\usepackage[T1]{fontenc} % Use 8-bit encoding that has 256 glyphs
	\linespread{1.05} % Aumentando o espaçamento entre as linhas (a fonte não fica tão legal com o espaçamento padrão)
	\usepackage{microtype} % Não lembro o que isso faz

% Configuração das fontes
\usepackage{fontspec}
	\setmainfont{TeX Gyre Pagella} % Palatino clone
	\setmonofont{BlexMonoNerdFontPropo}[Scale=0.75]
	% Configuração de fallback para emojis
	\newfontfamily{\emojifont}{Noto Color Emoji}[Renderer=HarfBuzz,Scale=0.75]
	\usepackage{newunicodechar}
	\newunicodechar{🐍}{{\emojifont 🐍}}
	\newunicodechar{🦀}{{\emojifont 🦀}}

\usepackage{enumitem} % Listas customizadas
	\setlist[itemize]{noitemsep} % Pra tornar as listas mais compactas

\usepackage{abstract}
	\renewcommand{\abstractnamefont}{\normalfont\bfseries} % Deixa o "Resumo" em negrito
	% \renewcommand{\abstracttextfont}{\normalfont\small\itshape} % Deixa o conteúdo do resumo em itálico
	\renewcommand{\abstracttextfont}{\normalfont\small}

\usepackage{titlesec} % Customização do título
	\renewcommand\thesection{\Roman{section}} % Números romanos para as secções
	\renewcommand\thesubsection{\roman{subsection}} % Para as subsecções também
	\titleformat{\chapter}[display]{\normalfont\huge\bfseries}{\chaptertitlename\ \thechapter}{20pt}{\Huge} % Formato dos capítulos
	\titlespacing*{\chapter}{0pt}{0pt}{20pt} % Reduz drasticamente a margem superior dos capítulos (era ~50pt por padrão)
	\titleformat{\section}[block]{\large\scshape}{\thesection.}{1em}{} % Muda a aparência do título das secções
	\titleformat{\subsection}[block]{\large}{\thesubsection.}{1em}{} % Muda a aparência do título das subsecções

\usepackage{fancyhdr} % Cabeçalho
	\pagestyle{fancy} % Cabeçalho em todas as páginas
	\fancyhead{}
	\fancyfoot{}
	\fancyhead[L]{Linguagens de Programação}% $\bullet$ Lucca Pellegrini $\bullet$ \today} % Custom header text
	\fancyhead[C]{\textbf{Rust}}% $\bullet$ Lucca Pellegrini $\bullet$ \today} % Custom header text
	\fancyhead[R]{PUC Minas}% $\bullet$ Lucca Pellegrini $\bullet$ \today} % Custom header text
	\fancyfoot[C]{\thepage} % Custom footer text

\usepackage{titling} % Customização do título

\usepackage{url} % Pra ajudar a lidar com urls chatos

\usepackage{amsmath, amsthm, amssymb, amsfonts} % AMS-TeX pra equações (em geral) mais bonitas e pra umas outras coisas
\usepackage{csquotes}
\usepackage{svg}

\usepackage{enotez}
	\setenotez{list-name={Notas}, backref=true}


\title{{\huge\bfseries Rust: Uma Análise da Linguagem}\\
Linguagens de Programação}

\author{ % Autores
    \textsc{Amanda Canizela Guimarães} \\
    \normalsize{\href{mailto:amanda.canizela@gmail.com}{\texttt{amanda.canizela@gmail.com}}}
    \and
    \textsc{Ariel Inácio Jordão Coelho} \\
    \normalsize{\href{mailto:arielijordao@gmail.com}{\texttt{arielijordao@gmail.com}}}
    \and
    \textsc{Lucca Pellegrini} \\
    \normalsize{\href{mailto:lucca@verticordia.com}{\texttt{lucca@verticordia.com}}}
}

\date{\today}

% =========================
% Documento
% =========================
\begin{document}

\maketitle

\begin{abstract}

    Esse trabalho aprofunda os conhecimentos sobre a linguagem de programação
    Rust. Nele, será apresentada a história de tal código, junto da criação de
    um panorama da mesma, colaborando para um melhor entendimento de suas
    aplicações e utilidades, tanto atuais quanto ao longo da história. Além
    disso, o grupo trará exemplos para realizar uma análise de sua
    implementação, mostrando o uso da linguagem e seus paradigmas diante de
    práticas cotidianas e mais avançadas. Por fim, será posto em evidência a
    importância da linguagem em questão para a fomentação de outras e seu
    impacto na história do desenvolvimento tecnológico.

\end{abstract}

% \chapter{Introdução}
%
% O primeiro parágrafo de cada seção não deve ser indentado. Este modelo segue as
% especificações oficiais da SBC para artigos submetidos.
%
% Os parágrafos subsequentes devem ter uma indentação de 1,27 cm no início da
% primeira linha, conforme as regras estabelecidas.
%
% \chapter{Panorama geral}
%
% Nesta seção será apresentada uma visão geral sobre a história da linguagem
% Rust, suas influências e o contexto de seu surgimento no cenário da programação
% moderna.
%
% \chapter{Aplicações e mercado}
%
% Aqui serão discutidas as principais áreas de aplicação da linguagem Rust, bem
% como sua presença no mercado de tecnologia e em projetos de código aberto.

\chapter{Classificação e Paradigmas}

\section{Introdução}

% TODO: revisar se essa introdução não repete muito do que será dito acima

Rust é uma linguagem compilada de propósito geral orientada ao desenvolvimento
de sistemas---isso é, orientada àquelas aplicações em que o desempenho do
programa é altamente valorizado---geralmente compreendida como uma alternativa
moderna e funcional a C++ que minimiza os riscos associados à gerência manual
de memória, por meio de uma variedade de abstrações, sem sacrificar desempenho
\cite[Prefácio]{blandy2021programming}. A linguagem segue a mesma ambição
proposta pelo autor de C++:

\begin{quote}
    Em geral, implementações de C++ seguem o princípio do
    \textit{zero-overhead}: você não paga pelo que não usa. E, além disso,
    aquilo que você usa não poderia ser implementado manualmente de forma
    melhor. \cite{stroustrup2004abstraction} (tradução livre)
\end{quote}

Apesar disso, o Rust não se limita apenas à programação de sistemas
operacionais, sistemas embarcados, e outras aplicações de \textit{baixo nível}:
suas características ergonômicas e sua flexibilidade possibilitam o
desenvolvimento de servidores web, ferramentas para \textit{DevOps}, interfaces
gráficas, aplicações de web \textit{back end} e---por meio de \textit{web
assembly}---\textit{front end}, jogos digitais, bancos de dados, compiladores,
aplicativos móveis, análise e transcodificação de multimídia, criptomoedas,
programas assíncronos, aplicações de bioinformática, ferramentas de busca,
sistemas \textsc{IoT}, aprendizado de máquina, navegadores, \textit{et cetera}.
Seu ecossistema, já amadurecido, inclui uma variedade de bibliotecas para os
desenvolvedores, além de um ferramental integrado, que incorpora servidores
\textit{Language Server Protocol} (\textsc{lsp}) para integração com
\textsc{ide}s, formatadores de código, e resolução automática e controle de
versões de dependências. \cite[Introdução]{klabnik2021rust}

\section{Sintaxe Básica}

A grosso modo, a linguagem dispõe de uma sintaxe familiar e \textit{C-like}, e
segue um paradigma \textit{imperativo e estruturado}, com certas
características procedurais clássicas, e outras funcionais modernas. A tipagem
é estática e extremamente rígida, para providenciar segurança a tempo de
compilação, mas o compilador é capaz de inferir o tipo de uma variável usando
\textit{inferência bidirecional} \cite{pierce2000local}. No exemplo abaixo,
usamos a palavra \mintinline{rust}{fn} para declarar uma função
\mintinline{rust}{main}, que não recebe parâmetros, e marca o ponto de entrada
do programa---nela, usamos a palavra \mintinline{rust}{let} para declarar uma
variável \texttt{x}, e em seguida o macro \mintinline{rust}{println!()} exibe o
valor dessa variável na saída padrão.\endnote{O ponto de exclamação indica que
\mintinline{rust}{println!()} não é uma função, mas um macro: no Rust, muitas
funcionalidades são implementadas por meio de metaprogramação com macros
sofisticados, que permitem que o compilador otimize ao máximo até mesmo as
operações mais simples. Nesse caso, o macro é executado a tempo de compilação,
e é o \textit{compilador} que gera o código que converte nosso inteiro para uma
string a ser exibida---ou seja, é muito mais eficiente que um
\texttt{printf()} em uma linguagem tradicional, que decide como fazer a
conversão a tempo de execução. O tópico de metaprogramação é extenso e
complexo, mas é interessante ressaltar que tudo ocorre a tempo de compilação: o
compilador compila o código-fonte do macro, gera uma biblioteca compartilhada,
carrega ela dinamicamente, e a usa para gerar o código de máquina apropriado
toda vez que o macro é referenciado. \cite[cap.~7]{gjengset2021rust}} O tipo de
\texttt{x} é inferido como \mintinline{rust}{i32}, um número inteiro de 32
bits, com sinal.\endnote{Há uma variedade de tipos escalares padrão, mas, em
geral, tipos numéricos têm tamanhos explícitos: \mintinline{rust}{u16} é um
\textit{unsigned} de 16 bits, \mintinline{rust}{u8} é um \textit{byte}
individual, \mintinline{rust}{f64} é um \textit{float} de 64 bits,
\mintinline{rust}{isize} é um \textit{signed} cujo tamanho é igual ao tamanho
de uma palavra na arquitetura atual, e assim por diante.}

\begin{minted}{rust}
    fn main() {
        let x = 5;
        println!("O valor de ‘x’ é: {x}");
    }
\end{minted}

Todas as variáveis são imutáveis por padrão. Para modificá-las, é preciso
declará-las mutáveis com a palavra \mintinline{rust}{mut}. Também é possível
especificar o tipo da variável após seu  nome na declaração, como se faz em
Python ou TypeScript. No exemplo abaixo, declaramos a variável \textit{mutável}
\texttt{c} com tipo \mintinline{rust}{char}---isso é, um caractere
\textsc{utf}-8.\endnote{Para ser mais exato, um \mintinline{rust}{char} é um
\textit{code point} especificado pelo padrão \textsc{utf}-8 \cite[§1]{rfc3629},
e não um \textit{grapheme cluster} do padrão Unicode \cite[§3]{uax29}. A
distinção é sutil, mas importante para internacionalização da linguagem.
\cite[cap.~3, §3]{blandy2021programming}}

\begin{minted}{rust}
    fn main() {
        let mut c: char = '🐍';
        println!("O valor de ‘c’ inicialmente é: {c}");
        c = '🦀'; // Modificar a variável só é possível pois usamos ‘mut’ acima.
        println!("O valor de ‘c’ modificado é: {c}");
    }
\end{minted}

% TODO: verificar se realmente falaremos sobre os diferentes tipos de string

Isso vale não somente para os tipos primitivos, mas também para os tipos
compostos. Há três tipos compostos padrão que podem estar na \textit{stack}:
tuplas, arrays, e slices. Uma tupla é um agrupamento composto de valores de
tipos variados com tamanho fixo---uma vez criada, o seu tamanho não pode mudar.
Elas podem ser declaradas listando seus itens, separados por vírgulas, entre
parênteses, e seus elementos podem ser acessados com um ponto (\texttt{.})
seguido de um índice numérico, ou por desestruturação. Já um array é um
agrupamento contíguo de elementos do mesmo tipo; seu tamanho também é fixo, e
seus elementos são acessados com um índice numérico entre colchetes, com
verificação do índice feita automaticamente para impedir acessos \textit{out of
bounds}. Uma slice, por outro lado, é um tipo de \textit{referência} (que em
Rust chamamos de \textit{empréstimo} ou \textit{borrow}) \cite[cap.~16,
§5]{google2024comp}, declarada prefixando o tipo com o caractere \texttt{\&},
que aponta para uma sequência contígua de elementos.\endnote{A fim de exemplo,
podemos declarar uma referência a um tipo primitivo assim:
\mintinline{rust}{let b: &i32 = &a}. Com isso, declaramos uma variável
\texttt{b}, de tipo \mintinline{rust}{&i32} (isso é, uma referência ou
\textit{empréstimo} de um inteiro de 32 bits com sinal), que aponta para uma
variável \texttt{a}, de tipo \mintinline{rust}{i32}. Por outro lado, se
declaramos \mintinline{rust}{let b: &[i32] = &a}, temos uma \textit{slice} de
um array de \mintinline{rust}{i32}s: um empréstimo de um segmento contíguo da
memória que contém zero ou mais elementos do tipo \mintinline{rust}{i32}. O
\textit{borrow checker} do Rust, que discutiremos mais adiante, assegura que o
empréstimo seja válido, e impede que a slice seja usada se o dado referenciado
deixar de existir---essa garantia é feita a tempo de compilação. Diferentemente
de uma referência simples, a slice também carrega informações de comprimento
(\texttt{len}) e capacidade.} No exemplo a seguir, uma tupla contendo um
inteiro, uma \textit{string slice},\endnote{Mais adiante, explicaremos os
diferentes tipos de strings em Rust, para esclarecer porque e quando usamos
\mintinline{rust}{String}, \mintinline{rust}{&str}, \mintinline{rust}{CStr},
\mintinline{rust}{OsStr}, \mintinline{rust}{OsString}, \mintinline{rust}{Path},
\mintinline{rust}{PathBuf}, \textit{et cetera}.} e um booleano é declarada,
acessada e desestruturada; abaixo, criamos um array mutável de
\textit{doubles}, alteramos um elemento, e depois criamos uma slice
não-mutável, que referencia os elementos da segunda à quarta posição no array
(índices 1 a 3):

\begin{minted}{rust}
    fn main() {
        // Criamos uma tupla, com três elementos de diferentes tipos
        let minha_tupla: (i32, &str, bool) = (47, "Hello, World!", false);
        println!("Item na posição 1 da tupla: {}", minha_tupla.1);

        // Desestruturamos a tupla, criando três variáveis individuais
        let (num, string, booleano) = minha_tupla;
        println!("Variáveis: ({}, {}, {})", num, string, booleano);

        // Criamos um array mutável com capacidade para cinco floats de 64 bits
        let mut meu_array: [f64; 5] = [0.1, 1.0, 1.5, 7.25, 1e100];
        meu_array[0] = 3.14;
        println!("Item no índice 0 do array: {}", meu_array[0]);

        // Criamos uma referência imutável que ‘enxerga’ os índices 1 a 3 do array
        let minha_slice: &[f64] = &meu_array[1..4];
        println!("Item no índice 2 da slice: {}", minha_slice[2]);
    }
\end{minted}

\section{Instruções como Expressões}

\subsection{Funções e Condicionais}

Em Rust, todo e qualquer bloco de código  delimitado por chaves, como o corpo
de uma função ou de um condicional, pode ser uma \textit{expressão}. Em outras
palavras, sempre que há um bloco com chaves, é possível \textit{retornar um
valor}, desde que apareça em um contexto onde um valor é esperado. Isso vale
para condicionais, laços com \mintinline{rust}{break} valorado, e blocos de
\textit{pattern matching}. Em geral, não é necessário usar a palavra
\mintinline{rust}{return}, a menos que desejemos retornar um valor que não está
\textit{ao final} do bloco; em geral, basta especificar a expressão a ser
retornada ao final do bloco, sem o ponto e vírgula. Um bloco que não contém um
\mintinline{rust}{return}, e que termina com ponto e vírgula, implicitamente
retorna um \mintinline{rust}{()}, que é o tipo vazio, ou \mintinline{c}{void},
do Rust.\endnote{Tecnicamente, o \mintinline{rust}{()} é o \textit{valor
unitário}---ele pertence ao \textit{tipo unitário}, que só tem um valor
possível, então geralmente não fazemos distinção entre o \textit{tipo} e o seu
\textit{valor}. Nos exemplos que mostramos acima, declaramos a função principal
como \mintinline{rust}{fn main() {}}, mas nada nos impede de explicitar o tipo
de retorno vazio com \mintinline{rust}{fn main() -> () {}}.} Por exemplo, se
quisermos implementar uma função recursiva para computar o fatorial de um
\mintinline{rust}{u32}, as três versões abaixo são sintaticamente
equivalentes.\endnote{Na verdade, como o Rust herda muito das linguagens
funcionais, a terceira abordagem, cuja sintaxe é mais limpa e
\textit{declarativa}, tende a ser preferida, e o ferramental integrado da
linguagem provavelmente reclamará se o seu projeto incluir funções que usam a
palavra \mintinline{rust}{return} desnecessariamente.} \cite[cap.~3,
§3]{klabnik2021rust}

\begin{minted}{rust}
    // Note que o tipo de retorno é indicado com uma seta, após os argumentos
    fn factorial(x: u32) -> u32 {
        if x <= 1 {
            return 1;
        } else {
            return x * factorial(x - 1);
        }
    }
\end{minted}

\begin{minted}{rust}
    fn factorial(x: u32) -> u32 {
        let result = if x <= 1 { 1 } else { x * factorial(x - 1) };
        return result;
    }
\end{minted}

\begin{minted}{rust}
    fn factorial(x: u32) -> u32 {
        if x <= 1 { 1 } else { x * factorial(x - 1) }
    }
\end{minted}

No terceiro exemplo, a expressão \mintinline{rust}{if}-\mintinline{rust}{else}
retorna um valor, que será determinado pela condição: se \mintinline{rust}{x <=
1}, o primeiro bloco é avaliado, e o \mintinline{rust}{1} será retornado, já
que é a única expressão nele, e não precede um ponto e vírgula---caso
contrário, o segundo bloco é avaliado. Pode-se entender, observando o
aninhamento dos blocos com chaves na função, que cada \textit{branch} da
condicional pode retornar um valor, a expressão condicional \textit{em si}
também retorna um valor, e a função, que só contém essa expressão condicional,
retornará o mesmo que ela. Note que, como os blocos que seguem os condicionais
\mintinline{rust}{if c1 {} else if c2 {} else {}} são expressões, as chaves são
obrigatórias, enquanto a condição em si é uma expressão simples, e não precisa
de parênteses como em outras linguagens \textit{C-like}---a obrigatoriedade das
chaves ao redor dos blocos é suficiente para desambiguar a condição do
consequente.

\subsection{Laços de Repetição}

Em Rust, há três tipos de laços de repetição: \mintinline{rust}{loop},
\mintinline{rust}{while} e \mintinline{rust}{for}. Desses, o laço infinito
\mintinline{rust}{loop} é o único que pode retornar um valor diferente de
\mintinline{rust}{()}, e só pode ser terminado com um
\mintinline{rust}{break}---os outros laços sempre podem ser reduzidos a esse,
mas existem por conveniência e legibilidade. Um \mintinline{rust}{break} pode
opcionalmente ser \textit{valorado}---isso é, pode receber um valor para o
\mintinline{rust}{loop} retornar---e também pode receber um \textit{rótulo},
para indicar \textit{de qual laço} o programa deve sair, quando há ambiguidade
entre laços aninhados. Um \mintinline{rust}{while} funciona como em qualquer
outra linguagem, repetindo o bloco até que sua condição seja falsa, e um
\mintinline{rust}{for} itera sobre os elementos de um valor iterável, como um
array, um vetor\endnote{Até então, só mencionamos tipos que vivem na stack, mas
a biblioteca padrão do Rust dispõe de uma variedade de tipos para dados no
heap, sempre seguindo o conceito de \textit{posse} e \textit{empréstimo}
(explicado adiante, e semelhante ao \textsc{raii} com \textit{move semantics}
em C++ moderno). O vetor \mintinline{rust}{Vec<T>}, do módulo
\mintinline{rust}{std::vec}, tem capacidade mutável e faz gerência automática
da memória. \cite[cap.~8, §1]{klabnik2021rust}
\cite[\stddoc{vec/struct.Vec.html\#capacity-and-reallocation}{\protect\mintinline{rust}{struct std::vec::Vec},
§4}]{rust_std_doc}}, um dicionário, \textit{et cetera}. No exemplo abaixo (que
não é o mais eficiente, mas ilustra bem o conceito), implementamos uma função
que recebe, por empréstimo, uma matriz de doubles---um vetor de vetores
\mintinline{rust}{Vec<Vec<f64>>}---busca um certo elemento \mintinline{rust}{x:
f64} nela, e retorna opcionalmente\endnote{O tipo \mintinline{rust}{Option<T>}
é um \textit{enum}, ou \textit{tipo de enumeração}, amplamente usado na
biblioteca padrão. Se um \mintinline{rust}{struct} pode ser entendido como uma
\textit{conjunção} de vários valores ou atributos agrupados, um
\mintinline{rust}{enum} pode ser entendido como a \textit{disjunção} de vários
valores possíveis. No caso do \mintinline{rust}{Option<T>}, ele só pode ter
dois valores: \mintinline{rust}{None} ou \mintinline{rust}{Some<T>}. Isso é,
ele representa \textit{nenhum valor} ou \textit{um valor de tipo
\mintinline{rust}{T}}. Usa-se no lugar de um \mintinline{java}{null} de
linguagens típicas, já que este introduz dificuldades graves de garantir a
segurança e a corretude do programa. \cite[cap.~6, §1.2]{klabnik2021rust}
\cite[\stddoc{option/index.html}{\protect\mintinline{rust}{mod std::option}}]{rust_std_doc}}
as coordenadas, se encontrado, como uma tupla \mintinline{rust}{(usize,
usize)}.\endnote{Usa-se \mintinline{rust}{usize} pois o índice de um vetor ou
de um array representa um deslocamento relativo a algum endereço em memória,
então seu tamanho depende da arquitetura. É semelhante ao uso de
\mintinline{c}{size_t} em C, apesar de que, na prática, programas em C
frequentemente optam por usar \mintinline{c}{int}s por convenção.}

\begin{minted}{rust}
    fn busca_matriz(matriz: &Vec<Vec<f64>>, x: f64) -> Option<(usize, usize)> {
        'loop_externo: loop {
            // O ‘enumerate’ produz o índice atual, junto ao elemento, em uma tupla
            for (i, linha) in matriz.iter().enumerate() { // Itera sobre linhas
                for (j, valor) in linha.iter().enumerate() { // Itera sobre colunas
                    // Se o nosso ‘x’ foi encontrado, saímos do loop externo
                    if *valor == x {
                        break 'loop_externo Some((i, j));
                    }
                }
            }

            // Se não encontrado, retornamos ‘None’
            break None; // O rótulo é desnecessário aqui, pois não há ambiguidade
        }
    }
\end{minted}

% \section{Análise crítica}
%
% Esta seção apresentará uma análise crítica da linguagem Rust, destacando três
% pontos positivos e três pontos negativos observados a partir de seu uso e das
% avaliações da comunidade.
%
% \section{Legado e influência}
%
% Será abordado o legado deixado pelo Rust, bem como as linguagens, ferramentas e
% práticas de desenvolvimento que foram influenciadas por ele.
%
% \section{Conclusão}
%
% A seção final sintetiza os principais pontos discutidos no trabalho, reforçando
% a relevância da linguagem Rust no cenário atual da computação.

\clearpage
\printendnotes

\clearpage
\printbibliography

\end{document}

% vim: et ts=4 sts=4 sw=4
